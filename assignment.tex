\documentclass{article}
\usepackage{assignmentstyle}

\begin{document}
	\maketitle
	\pagestyle{fancy}
	\thispagestyle{empty}

	\part{Introduction}
	
	\question{What is AI?}

	\answer

	Artificial intelligence is a new science for stimulating and extending the intelligence of man and the relative theories, methods, technologies and implemetary systems.

	The purpose of AI research is to prompt machine abilities in listening, recognizing, speaking, thinking and acting. 

	The definition varies to different criteria; researches are focuesed either on thought processes and reasoning, or on behavior and action. 

	\question{About Turing test?}

	\answer

	The test questions whether machines can behave intelligently. It is an imitation game where AI system tries to interact like a human through computer interaction. If the tester recognizes the component as human, the AI system actually passes the test.

	\question{Expert System?}

	\answer

	An expert system is a computer system that emulates, or acts in all respects, with the decision-making capabilities of a human expert.

	\question{AI schools?}

	Logic/Symbol and Data/Connectionism.

	\part{Predicate Calculus(propositional logic)}

	\question{Six necessary points for AI system?}

	\answer

	\begin{enumerate}
		\item Auto learning;
		\item Made to solve heuristic problems rather than computational problems;
		\item Able to handle uncertain information;
		\item Able to handle qualitative problems;
		\item Transform from problems to machine tasks;
		\item Specialized for field knowledge and specific problems;
	\end{enumerate}

	\sep

	\question{What is knowledge?}

	\answer
	
	In Epistemology, knowledge is a philosophic proposition, which can be classified into priori knowledge and posteriori knowledge.

	\begin{enumerate}
		\item posteriori knowledge, derived from the senses; not always reliable; denialble on the basis of new knowledge;
		\item priori knowledge, independent of the senses; universally true, cannot be denied without contradiction;
	\end{enumerate}

	Knowledge can also be divided into three parts: 

	\begin{enumerate}
		\item Procedural knowledge, knowing how to do something. e.g., fix a watch.
		\item Declarative knowledge, knowing something to be true or false. e.g., time is expensive.
		\item Tacit knowledge, unconscious knowledge; cannot be expressed by language. e.g., knowing how to walk, breathe.
	\end{enumerate}
	
	\sep
	
	\question{Knowledge, data, information?}

	\answer

	\begin{description}
		\item[Wisdom] Using knowledge in a beneficial way
		\item[Metaknowledge] Rules about knowledge
		\item[Knowledge] Rules about using information
		\item[information] Potentially useful for knowledge
		\item[Data] Potentially useful information
		\item[Noise] No apparent information     
	\end{description}

	\sep

	\question{Metaknowledge in AI?}

	\answer

	\begin{enumerate}
		\item An ontology is the metaknowledge that describes everything known about the problem domain.
		\item How to control and search knowledge.
	\end{enumerate}

	\sep
	
	\question{Core and goal of knowledge-based system?}

	\answer

	Programs = Algorithms + Data Structures

	Knowledge-based System = Inference + Knowledge

	Sparate actual meanings of words with reasoning process itself, make inferences w/o relying on semantics, and reach valid conclusions based on facts only.

	\sep

	\question{Statement for implication($p\rightarrow q$)?}

	\answer

	\begin{enumerate}
		\item if p then q
		\item q if p
		\item p implies q
		\item p is sufficient for q
		\item q is necessary for p
		\item p only if q
	\end{enumerate}

	\sep

	\notice{$p\leftrightarrow q$ is true if p and q are both true or both false.}

	\sep

	\question{Other logic symbol and definition}

	\answer

	\begin{description}
		\item[Tautology] a statement that is always true
		\item[Contradiction] a statement that is always false
		\item[Contingent] a statement that is neither a tautology nor a contradiction
		\item[Implication] $p\Rightarrow q$ If a conditional is a tautology, it is an implication.
		\item[Logical Equivalence] $p\Leftrightarrow \lnot\lnot p$ or $p\equiv \lnot\lnot p$ If a biconditional is a tautology, it is a logical equivalence or material equivalence.
	\end{description}

	\sep
	
	\question{Well-formed formulas(wff's)?}

	\answer

	\begin{enumerate}
		\item Proposition symbols(numbers, alphabets; first letter upper-classed)
		\item Connectives
		\item Truth symbols
	\end{enumerate}

	e.g., P1$\equiv$True.

	\sep

	\part{Predicate Calculus(first-order predicate logic)}

	\question{Shortages of propositional logic?}

	\answer

	Propositional logic can only deal with \textbf{complete} statements; it cannot validate internel structures of statements or syllogisms.

	\sep
	
	\question{Simple predicate?}

	\answer

	\begin{enumerate}
		\item name, lowercase letter followed by numbers/alphabets/underlines
		\item arguments, \textbf{lowercase} letter followed by numbers/alphabets/underlines called \textbf{constant or function}; \textbf{uppercase} letter followed by ... called \textbf{variables}
	\end{enumerate}

	\sep
	
	\notice{In existential quantifier, always be alert to use "$\rightarrow$", as predicate relationshop may not always be sound. e.g., $$(\exists X)(\text{elephant}(X)\land \text{three\_legged}(X))$$}

	\notice{Scope of quantifier is over all variables in the brackets. e.g., $$(\forall X)(\text{a}(X)\land(\exists Y)(\text{b}(Y)\lor\text{c}(Y)))$$which means for all $X$, there are some $Y$ satisfies the prodicate statement.}

	\sep

	\question{Extract negated quantifiers?}

	\answer

	$$\lnot (\forall X)(\text{p}(X))\equiv(\exists X)(\lnot \text{p}(X))$$

	$$\lnot (\exists X)(\text{p}(X))\equiv(\forall X)(\lnot \text{p}(X))$$

	\sep

	\question{Turn syllogistic logic into predicate logic?}

	\answer

	\begin{description}
		\item[universal affirmative] "All $S$ is $P$." $(\forall X)(\text{s}(X)\rightarrow\text{p}(X))$
		\item[universal negative] "No $S$ is $P$." $(\forall X)(\text{s}(X)\rightarrow\lnot\text{p}(X))$
		\item[particular affirmative] "Some $S$ is $P$." $(\exists X)(\text{s}(X)\land\text{p}(X))$
		\item[particular negative] "Some $S$ is not $P$." $(\exists X)(\text{s}(X)\land\lnot\text{p}(X))$  
	\end{description}

	\sep

	\question{Rules of inferences?}

	\answer

	\begin{table}[H]
		\centering
		\begin{tabular}{ll}
			\toprule
			modus ponens & $(P\rightarrow Q)\land P\Rightarrow Q$ \\
			modus tollens & $(P\rightarrow Q)\land \lnot Q\Rightarrow \lnot P$ \\
			chains rule & $(P\rightarrow Q)\land(Q\rightarrow R)\Rightarrow P\rightarrow R$ \\
			\midrule
			\multirow{3}{*}{basic equivalences} & $P\rightarrow Q\equiv \lnot P\lor Q$ \\
			& $P\land\lnot P\equiv F$ \\
			& $P\lor \lnot P\equiv T$ \\
			\midrule
			\multirow{2}{*}{the commutative law} & $P\land Q\equiv Q\land P$ \\
			& $P\lor Q\equiv Q\lor P$ \\
			\midrule
			\multirow{2}{*}{the distribution law} & $P\land (Q\lor C)\equiv (P\land Q)\lor(P\land C)$ \\
			& $P\lor (Q\land C)\equiv (P\lor Q)\land(P\lor C)$ \\
			\midrule
			\multirow{2}{*}{the associative law} & $P\land (Q\land C)\equiv(P\land Q)\land C$ \\
			& $P\lor (Q\lor C)\equiv(P\lor Q)\lor C$ \\
			\midrule
			\multirow{2}{*}{De Morgan's law} & $\lnot(P\land Q)\equiv\lnot P\lor\lnot Q$ \\
			& $\lnot(P\lor Q)\equiv\lnot P\land\lnot Q$ \\
			\midrule
			the contrapositive law & $P\rightarrow Q\equiv \lnot Q\rightarrow\lnot P$ \\
			\midrule
			the double negative law & $\lnot\lnot P\equiv P$ \\
			\midrule
			\multirow{2}{*}{the absorbtive law} & $P\lor(P\land Q)\equiv P$ \\
			& $P\land(P\lor Q)\equiv P$ \\
			\toprule
		\end{tabular}
	\end{table}
\end{document}
